\section{So sánh và đánh giá hiệu suất các mô hình}

Trong khuôn khổ của nghiên cứu này, bốn mô hình phân loại đã được nhóm em xây dựng và đánh giá nhằm mục tiêu dự đoán khả năng mắc bệnh tim (\texttt{HeartDisease}) dựa trên các yếu tố lâm sàng. Các mô hình bao gồm:
\begin{enumerate}
    \item Hồi quy Logistic (Logistic Regression) - Mô hình cơ sở (Baseline)
    \item K-Nearest Neighbors (KNN) - Mô hình dựa trên khoảng cách
    \item Cây Quyết định (Decision Tree) - Mô hình dựa trên quy tắc (diễn giải được)
    \item Rừng ngẫu nhiên (Random Forest) - Mô hình Ensemble
\end{enumerate}
Tất cả các mô hình đều được tinh chỉnh siêu tham số (hyperparameter tuning) bằng kỹ thuật \texttt{GridSearchCV} (với 5-fold cross-validation) để tìm ra bộ tham số tốt nhất. Phần này nhóm em sẽ trình bày so sánh chi tiết hiệu suất của các mô hình (sau khi đã tinh chỉnh) dựa trên các chỉ số đánh giá phân loại phổ biến: \textbf{Accuracy}, \textbf{Precision}, \textbf{Recall}, và \textbf{F1-Score}, được tính toán trên tập dữ liệu kiểm tra (test set).

\subsection{Tổng hợp kết quả đánh giá hiệu suất}
\label{sec:tong-hop-ket-qua}

Bảng \ref{tab:so-sanh-all} tóm tắt các chỉ số hiệu suất chính của từng mô hình (đã được tinh chỉnh). Giá trị F1-Score được sử dụng làm thước đo so sánh chính vì nó cân bằng giữa Precision và Recall, đặc biệt quan trọng trong các bài toán y tế.

\begin{table}[h!]
    \centering
    \caption{Bảng so sánh chi tiết các chỉ số đánh giá hiệu suất của các mô hình phân loại (đã tinh chỉnh) trên tập kiểm tra.}
    \label{tab:so-sanh-all}
    \begin{tabular}{l c c c c}
        \toprule
        \textbf{Tên Mô hình} & \textbf{Accuracy} & \textbf{Precision} & \textbf{Recall} & \textbf{F1-Score} \\
        \midrule
        Hồi quy Logistic (Baseline) & 0.8370 & 0.8969 & 0.8131 & 0.8530 \\
        K-Nearest Neighbors (Tuned) & 0.8804 & 0.8972 & 0.8972 & 0.8972 \\
        Cây Quyết định (Tuned) & 0.8696 & 0.8879 & 0.8879 & 0.8879 \\
        \textbf{Rừng ngẫu nhiên (Tuned)} & \textbf{0.8750} & \textbf{0.8962} & \textbf{0.8879} & \textbf{0.8920} \\
        \bottomrule
    \end{tabular}
\end{table}
\subsection{Phân tích chi tiết và thảo luận Kết quả}
\label{sec:phan-tich-ket-qua}
Từ Bảng \ref{tab:so-sanh-all}, nhóm em tiến hành phân tích sâu hơn về hiệu suất của từng mô hình:

\subsubsection{Hồi quy Logistic (Logistic Regression)}
Mô hình Hồi quy Logistic đạt được \textbf{Accuracy = 83.70\%}, với \textbf{Precision = 89.69\%}, \textbf{Recall = 81.31\%}, và \textbf{F1-Score = 0.8530}. Đây là một mô hình cơ sở (baseline) vững chắc, cho thấy mối quan hệ giữa các đặc trưng và khả năng mắc bệnh tim có tính chất tuyến tính đáng kể. Precision cao (89.69\%) cho thấy mô hình ít cảnh báo nhầm (FP thấp), tuy nhiên Recall tương đối thấp hơn (81.31\%) có nghĩa là mô hình bỏ sót khoảng 19\% ca bệnh thực tế -- một điểm cần cải thiện trong bối cảnh y tế.

\subsubsection{K-Nearest Neighbors (KNN)}
Mô hình KNN, sau khi được tinh chỉnh tối ưu qua GridSearchCV, đạt kết quả ấn tượng với \textbf{Accuracy = 88.04\%}, \textbf{Precision = 89.72\%}, \textbf{Recall = 89.72\%}, và \textbf{F1-Score = 0.8972}. Đây là mô hình có hiệu suất cao nhất trong ba mô hình đầu tiên, vượt trội hơn hẳn baseline Logistic Regression. Đặc biệt, sự cân bằng hoàn hảo giữa Precision và Recall (cả hai đều 89.72\%) cho thấy mô hình đạt được trade-off tối ưu -- vừa phát hiện tốt ca bệnh (Recall cao), vừa ít cảnh báo giả (Precision cao). Biểu đồ Elbow Method (Hình \ref{fig:knn-elbow}) cho thấy error rate ổn định ở vùng $k \in [10, 20]$, và GridSearch đã tìm ra tổ hợp tham số tối ưu với metric khoảng cách phù hợp.

\subsubsection{Cây Quyết định (Decision Tree)}
Mô hình Cây Quyết định sau khi được tinh chỉnh và ``cắt tỉa'' (pruning) qua GridSearchCV đạt \textbf{Accuracy = 86.96\%}, \textbf{Precision = 88.79\%}, \textbf{Recall = 88.79\%}, và \textbf{F1-Score = 0.8879}. Đây là kết quả ấn tượng, cho thấy việc cắt tỉa đã giúp mô hình cân bằng tốt giữa Precision và Recall (cả hai đều 88.79\%). Hiệu suất này cao hơn đáng kể so với Logistic Regression và chỉ thấp hơn KNN và RF khoảng 1\%. 

\textbf{Ưu điểm vượt trội:} Mặc dù không phải mô hình có F1-Score cao nhất, Decision Tree có giá trị lớn về \textit{khả năng diễn giải} (interpretability). Mô hình cung cấp các quy tắc quyết định (decision rules) rõ ràng, dễ hiểu và có thể giải thích cho bác sĩ lâm sàng -- điều mà các mô hình "hộp đen" như KNN không thể làm được.

\textbf{Phân tích Feature Importance:} Hình \ref{fig:dt-importance} xác định \texttt{ST\_Slope\_Up} là yếu tố quan trọng nhất với tỷ trọng gần 60\%, tiếp theo là \texttt{Cholesterol}, \texttt{ExerciseAngina\_Y}, và \texttt{Sex\_M}. Cấu trúc cây (Hình \ref{fig:dt-tree}) cho thấy quy tắc phân loại đầu tiên chính là kiểm tra giá trị \texttt{ST\_Slope\_Up}, khẳng định tầm quan trọng của đặc trưng này trong chẩn đoán.

\textbf{Hiệu quả cắt tỉa:} So sánh với mô hình mặc định (F1-Train = 1.0000, F1-Test = 0.8756), mô hình sau tuning đạt F1-Train = 0.8826 và F1-Test = 0.8879, cho thấy việc cắt tỉa đã giảm overfitting thành công và cải thiện khả năng tổng quát hóa (F1-Test tăng từ 0.8756 lên 0.8879 -- tăng ~1.4\%).

\subsubsection{Rừng ngẫu nhiên (Random Forest)}
Mô hình Rừng ngẫu nhiên đạt \textbf{Accuracy = 87.50\%}, \textbf{Precision = 89.62\%}, \textbf{Recall = 88.79\%}, và \textbf{F1-Score = 0.8920}. Đây là mô hình có F1-Score \textit{thấp hơn KNN một chút} (0.8920 vs 0.8972), nhưng vẫn thể hiện hiệu suất ổn định và cân bằng tốt. Random Forest có Recall thấp hơn KNN (88.79\% vs 89.72\%), nghĩa là bỏ sót thêm khoảng 1\% ca bệnh, tuy nhiên vẫn vượt trội hơn hẳn Logistic Regression (81.31\%). Kết quả này khẳng định sức mạnh của phương pháp ensemble: bằng cách kết hợp nhiều cây quyết định (100-200 cây), Random Forest giảm được overfitting và phương sai (variance) so với Decision Tree đơn lẻ, đồng thời cải thiện đáng kể khả năng tổng quát hóa (F1 tăng từ 0.8744 lên 0.8920 -- tăng ~2\%). Feature Importance từ Random Forest (Hình \ref{fig:rf-importance}) củng cố kết luận của Decision Tree: \texttt{ST\_Slope\_Up} vẫn là yếu tố quan trọng nhất, tiếp theo là \texttt{Oldpeak}, \texttt{ST\_Slope\_Flat}, \texttt{MaxHR}, và \texttt{Cholesterol}.

\subsection{Thảo luận chung và Kết luận (Decision Making)}
\label{sec:ket-luan-so-bo}
Dựa trên các kết quả đánh giá, nhóm em rút ra các kết luận sau:

\begin{enumerate}
    \item \textbf{Về mặt hiệu suất:} Dựa trên F1-Score (thước đo chính), thứ hạng các mô hình từ cao đến thấp là:
    \begin{itemize}
        \item \textbf{1. K-Nearest Neighbors (F1 = 0.8972)} -- Mô hình tốt nhất với sự cân bằng hoàn hảo giữa Precision và Recall.
        \item \textbf{2. Random Forest (F1 = 0.8920)} -- Rất gần KNN, ổn định và ít overfitting.
        \item \textbf{3. Decision Tree (F1 = 0.8879)} -- Hiệu suất tốt, chỉ thấp hơn RF 0.4\%, và có khả năng diễn giải vượt trội.
        \item \textbf{4. Logistic Regression (F1 = 0.8530)} -- Baseline vững chắc nhưng Recall thấp nhất (81.31\%).
    \end{itemize}
    \textbf{Nhận xét quan trọng:} Khoảng cách giữa top 3 mô hình rất nhỏ (chỉ ~1\%), cho thấy KNN, RF và DT đều là lựa chọn khả thi. Decision Tree có lợi thế đặc biệt về \textit{tính diễn giải} -- quan trọng trong y tế khi bác sĩ cần hiểu ``tại sao'' mô hình dự đoán như vậy. KNN và RF tuy có F1 cao hơn chút ít nhưng là ``hộp đen'' (black box), khó giải thích quyết định.
    
    \item \textbf{Về mặt Khai phá Dữ liệu (Decision Making):} Đây là kết luận quan trọng nhất của dự án. Thông qua việc Trực quan hóa Dữ liệu (Chương 5) và phân tích \textbf{Feature Importance} từ các mô hình Cây Quyết định (Hình \ref{fig:dt-importance}) và Rừng ngẫu nhiên (Hình \ref{fig:rf-importance}), nhóm có thể đưa ra các kết luận (ra quyết định) về bài toán:
    \begin{itemize}
        \item Các yếu tố rủi ro quan trọng nhất ảnh hưởng đến bệnh tim (trong bộ dữ liệu này) đã được xác định.
        \item Đặc trưng \textbf{\texttt{ST\_Slope}} (độ dốc ST) liên tục được cả hai mô hình cây xác định là yếu tố dự đoán số một.
        \item Các yếu tố quan trọng tiếp theo bao gồm \textbf{\texttt{ChestPainType}} (loại đau ngực), \textbf{\texttt{MaxHR}} (nhịp tim tối đa), và \textbf{\texttt{Oldpeak}}.
    \end{itemize}
\end{enumerate}

Kết quả này nhấn mạnh rằng, để đưa ra quyết định sàng lọc bệnh tim, các mô hình học máy có thể hỗ trợ hiệu quả bằng cách tập trung vào các chỉ số lâm sàng có ảnh hưởng lớn nhất.