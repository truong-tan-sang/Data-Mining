\section{Giới thiệu dự án}

Trong bối cảnh y tế hiện đại, bệnh lý tim mạch (CVDs) là một trong những nguyên nhân gây tử vong hàng đầu trên toàn thế giới, đặt ra gánh nặng nghiêm trọng cho hệ thống y tế và xã hội. Việc phát hiện và chẩn đoán sớm bệnh tim đóng vai trò then chốt trong việc cải thiện tiên lượng cho bệnh nhân, giảm tỷ lệ tử vong và tối ưu hóa chi phí điều trị. Tuy nhiên, việc chẩn đoán sớm đòi hỏi phân tích nhiều yếu tố rủi ro phức tạp như tuổi tác, chỉ số cholesterol, huyết áp, và các chỉ số lâm sàng khác.

Do đó, đề tài \textbf{"Phân tích các yếu tố rủi ro ảnh hưởng tới Bệnh tim và Xây dựng mô hình Phân loại"} mang tính cần thiết và cấp thiết trong thực tiễn. Bằng việc áp dụng các phương pháp Khai phá Dữ liệu và Học máy hiện đại, đề tài hướng tới mục tiêu không chỉ tìm hiểu mối quan hệ giữa các yếu tố rủi ro mà còn xây dựng mô hình phân loại (classification) có độ chính xác cao, hỗ trợ các chuyên gia y tế trong việc sàng lọc và đưa ra quyết định kịp thời.

Để đạt được mục tiêu đó, nhóm đề tài sử dụng các phương pháp phân loại phổ biến trong lĩnh vực học máy, bao gồm:

\begin{itemize}
    \item \textbf{Hồi quy Logistic (Logistic Regression):} Mô hình tuyến tính cơ sở, hiệu quả cho bài toán phân loại nhị phân và cung cấp một baseline mạnh mẽ.
    
    \item \textbf{Cây quyết định (Decision Tree):} Phương pháp phân loại dữ liệu theo dạng cây, dễ diễn giải và trực quan hóa, rất quan trọng cho việc ra quyết định(decision making). 
    
    \item \textbf{Rừng ngẫu nhiên (Random Forest):} Mô hình ensemble dựa trên cây, có khả năng xử lý tốt các mối quan hệ phức tạp và giảm thiểu hiện tượng overfitting.
    
    \item \textbf{K-nearest neighbors (KNN):} Phương pháp phân loại dựa trên khoảng cách (instance-based), hiệu quả trong việc tìm ra các mẫu cục bộ trong không gian đặc trưng.
\end{itemize}

Với cách tiếp cận đa phương pháp như trên, đề tài kỳ vọng sẽ đưa ra được đánh giá toàn diện về các yếu tố ảnh hưởng và lựa chọn được mô hình dự đoán bệnh tim phù hợp nhất với dữ liệu thực tế.
\newpage