    \section{Tổng quan công việc}

\subsection{Giới thiệu về phương pháp luận}
Trong dự án này, chúng tôi áp dụng phương pháp luận CRISP-DM (Cross-Industry Standard Process for Data Mining) để thực hiện quá trình khai thác dữ liệu. Phương pháp này bao gồm 6 giai đoạn chính:
\begin{itemize}
    \item \textbf{Hiểu biết về nghiệp vụ (Business Understanding):} Xác định mục tiêu và yêu cầu của dự án (dự đoán khả năng mắc bệnh tim).
    \item \textbf{Hiểu biết về dữ liệu (Data Understanding):} Thu thập và phân tích dữ liệu ban đầu (bộ dữ liệu \texttt{heart.csv}).
    \item \textbf{Chuẩn bị dữ liệu (Data Preparation):} Tiền xử lý, làm sạch, mã hóa và chuẩn hóa dữ liệu.
    \item \textbf{Mô hình hóa (Modeling):} Xây dựng và đánh giá các mô hình (Logistic Regression, KNN, Decision Tree, Random Forest).
    \item \textbf{Đánh giá (Evaluation):} Đánh giá kết quả và kiểm tra mức độ đáp ứng mục tiêu của các mô hình.
    \item \textbf{Triển khai (Deployment):} Giai đoạn này nằm ngoài phạm vi của bài tập lớn.
\end{itemize}

\subsection{Phương pháp tiếp cận}
Dự án sử dụng phương pháp tiếp cận dựa trên dữ liệu (Data-Driven Approach) để phân tích và dự đoán khả năng mắc bệnh tim. Các bước thực hiện bao gồm:
\begin{itemize}
    \item \textbf{Thu thập dữ liệu:} Sử dụng bộ dữ liệu \texttt{heart.csv} được cung cấp.
    \item \textbf{Tiền xử lý dữ liệu:} Xử lý dữ liệu trùng lặp (\texttt{drop\_duplicates}), mã hóa One-Hot Encoding cho các biến phân loại và chuẩn hóa \texttt{StandardScaler} cho các biến số (đối với các mô hình nhạy cảm với thang đo như Logistic Regression và KNN).
    \item \textbf{Phân tích dữ liệu:} Phân tích thống kê mô tả, trực quan hóa ma trận tương quan và phân bố dữ liệu để hiểu rõ các đặc trưng.
    \item \textbf{Xây dựng mô hình:} Lựa chọn, huấn luyện và tinh chỉnh bốn thuật toán phân loại (Logistic Regression, KNN, Decision Tree, Random Forest) để tìm ra mô hình dự đoán tốt nhất.
\end{itemize}

\subsection{Các kỹ thuật sử dụng}
Dự án áp dụng các kỹ thuật khai thác dữ liệu sau:
\begin{itemize}
    \item \textbf{Phân tích thống kê:}
    \begin{itemize}
        \item Thống kê mô tả (thông tin cơ bản, heatmap tương quan).
        \item Trực quan hóa phân bố dữ liệu (ví dụ: boxplot).
    \end{itemize}
    \item \textbf{Học máy (Machine Learning):}
    \begin{itemize}
        \item \textbf{Học có giám sát (Supervised Learning):} Sử dụng cho bài toán phân loại nhị phân (dự đoán \texttt{HeartDisease} = 0 hoặc 1).
        \item \textbf{Các thuật toán:} Logistic Regression (Baseline), K-Nearest Neighbors (KNN), Decision Tree, và Random Forest (Ensemble).
    \end{itemize}
    \item \textbf{Xử lý dữ liệu:}
    \begin{itemize}
        \item Mã hóa One-hot (\texttt{pd.get\_dummies}) cho các biến categorical.
        \item Chuẩn hóa dữ liệu (\texttt{StandardScaler}) cho các biến numerical.
    \end{itemize}
    \item \textbf{Trực quan hóa dữ liệu:}
    \begin{itemize}
        \item Biểu đồ ma trận nhầm lẫn (Confusion Matrix heatmap).
        \item Biểu đồ phương pháp khuỷu tay (Elbow Method) để tìm $k$ tối ưu cho KNN.
        \item Trực quan hóa cây quyết định (\texttt{plot\_tree}).
    \end{itemize}
\end{itemize}

\subsection{Đánh giá và kiểm chứng}
Quá trình đánh giá và kiểm chứng được thực hiện thông qua:
\begin{itemize}
    \item \textbf{Phân chia dữ liệu:}
    \begin{itemize}
        \item Tập huấn luyện (training set): 80\% dữ liệu.
        \item Tập kiểm tra (test set): 20\% dữ liệu (\texttt{random\_state=42}).
    \end{itemize}
    \item \textbf{Đánh giá mô hình:}
    \begin{itemize}
        \item Độ chính xác (Accuracy).
        \item Độ chính xác (Precision).
        \item Độ nhạy (Recall).
        \item F1-score (chỉ số cân bằng giữa Precision và Recall).
    \end{itemize}
    \item \textbf{Kiểm chứng chéo và Tinh chỉnh:}
    \begin{itemize}
        \item Sử dụng \texttt{GridSearchCV} với kiểm chứng chéo 5-fold (cv=5) để tìm các siêu tham số (hyperparameters) tối ưu cho mỗi mô hình, tập trung vào việc tối ưu hóa chỉ số \texttt{recall} hoặc \texttt{f1-score}.
    \end{itemize}
\end{itemize}

\subsection{Tính mới và đóng góp}
Dự án này tập trung vào việc áp dụng và so sánh một cách có hệ thống bốn mô hình phân loại phổ biến (từ tuyến tính, dựa trên khoảng cách, đến ensemble) trên cùng một bộ dữ liệu dự đoán bệnh tim.
\begin{itemize}
    \item \textbf{Đóng góp chính:} Cung cấp một phân tích so sánh chi tiết về hiệu suất, ưu nhược điểm của từng thuật toán (Logistic Regression, KNN, Decision Tree, Random Forest) trong bối cảnh cụ thể của bài toán y tế này.
    \item \textbf{Tinh chỉnh mô hình:} Báo cáo này không chỉ xây dựng mô hình mặc định mà còn thực hiện tinh chỉnh siêu tham số (hyperparameter tuning) bằng \texttt{GridSearchCV} để tối ưu hóa hiệu suất, đặc biệt là các chỉ số quan trọng như Recall và F1-Score, vốn rất có ý nghĩa trong chẩn đoán y khoa.
\end{itemize}
\newpage