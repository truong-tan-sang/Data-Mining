\section{Cơ sở lý thuyết}

\subsection{Tổng quan về học máy và phân lớp dữ liệu}
Học máy (Machine Learning) là một nhánh của trí tuệ nhân tạo (AI), cho phép máy tính học từ dữ liệu và cải thiện hiệu suất dự đoán mà không cần lập trình một cách cụ thể. Trong bối cảnh dự đoán bệnh tim, học máy đóng vai trò quan trọng trong việc xây dựng các mô hình có khả năng khai thác các mối quan hệ phức tạp giữa nhiều chỉ số y tế khác nhau.

Học máy được chia thành nhiều loại, trong đó phổ biến nhất là:
\begin{itemize}
    \item \textbf{Học có giám sát (Supervised Learning):} Mô hình được huấn luyện trên tập dữ liệu có nhãn, tức là mỗi mẫu dữ liệu đều đi kèm với kết quả đầu ra mong muốn. Đây là phương pháp chính được sử dụng trong bài toán này.
    \item \textbf{Học không giám sát (Unsupervised Learning):} Tập trung vào việc tìm kiếm cấu trúc ẩn trong dữ liệu không có nhãn (ví dụ: phân cụm)
    \item \textbf{Học tăng cường (Reinforcement Learning):} Mô hình học thông qua tương tác với môi trường và tối ưu hóa phần thưởng.
\end{itemize}

Trong khuôn khổ bài tập lớn này, bài toán chính được xác định là một bài toán \textbf{phân loại nhị phân (binary classification)}, trong đó đầu ra cần dự đoán là một giá trị rời rạc (bệnh nhân có khả năng bị bệnh tim (\texttt{HeartDisease=1}) hay không (\texttt{HeartDisease=0})).

\subsection{Các thuật toán học máy}
Trong quá trình xây dựng mô hình dự đoán bệnh tim, việc lựa chọn thuật toán phù hợp là yếu tố then chốt để đảm bảo độ chính xác và khả năng tổng quát hóa của mô hình. Dưới đây là các thuật toán được sử dụng trong bài toán này. 

\subsubsection{Hồi quy Logistic (Logistic Regression)}
Đây là mô hình phân loại tuyến tính đơn giản, nhanh và dễ hiểu. Mô hình này sẽ đóng vai trò làm \textbf{baseline model} (mô hình cơ sở) để đo lường hiệu quả của các thuật toán phức tạp hơn sau này.
\subsubsection{K-Nearest Neighbors (KNN)}
KNN là một thuật toán học có giám sát \textit{dựa trên thể hiện (instance-based)}. Thuật toán này không tạo ra một hàm số cụ thể mà lưu trữ toàn bộ tập huấn luyện. Khi dự đoán, nó tìm $k$ điểm dữ liệu gần nhất trong tập huấn luyện và dự đoán dựa trên \textit{phiếu bầu đa số (majority vote)} của các điểm đó.

\subsubsection{Cây quyết định (Decision Tree)}
Decision Tree hoạt động bằng cách liên tục chia dữ liệu thành các nhóm nhỏ hơn dựa trên các phép toán điều kiện về đặc trưng (ví dụ: \texttt{Age < 50?}). Nó tìm ra câu hỏi tốt nhất (giúp phân tách \texttt{target=0} và \texttt{target=1} rõ nhất) tại mỗi bước. Mô hình này rất mạnh mẽ nhưng có một rủi ro lớn: \textbf{overfitting} (học quá khớp).

\subsubsection{Rừng ngẫu nhiên (Random Forest)}
Rừng ngẫu nhiên là một thuật toán học máy dạng ensemble (tập hợp), được sử dụng cho cả bài toán phân loại và hồi quy. Thuật toán này xây dựng nhiều cây quyết định (decision trees) trong quá trình huấn luyện và cho ra kết quả dự đoán bằng cách... bỏ phiếu số đông (đối với phân loại) từ các cây. Ý tưởng chính... là bằng cách kết hợp nhiều cây yếu (weak learners), mô hình tổng thể sẽ đạt được hiệu suất tốt hơn và hạn chế được hiện tượng quá khớp (overfitting).

Nó chống overfitting bằng 2 kỹ thuật chính:
\begin{enumerate}
    \item \textbf{Bagging (Bootstrap Aggregating):} Mỗi cây được huấn luyện trên một mẫu \textit{con} ngẫu nhiên (lấy có lặp lại) từ tập dữ liệu huấn luyện.
    \item \textbf{Feature Randomness:} Tại mỗi nút (node) của cây, thay vì xem xét \textit{tất cả} các đặc trưng, cây chỉ được phép chọn ngẫu nhiên một \textit{tập con} các đặc trưng... để tìm ra phép chia tốt nhất.
\end{enumerate}
\newpage